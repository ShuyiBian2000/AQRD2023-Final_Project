% Options for packages loaded elsewhere
\PassOptionsToPackage{unicode}{hyperref}
\PassOptionsToPackage{hyphens}{url}
\documentclass[12pt, ]{article}

\usepackage{mathtools}
\usepackage{amsmath}
\usepackage{amsthm}
\usepackage{amssymb}
\usepackage[italicdiff]{physics}
\mathtoolsset{showonlyrefs}

% SPACING AND FONTS %%%%%%%%%%%%%%%%%%%%%%%%%%%%%%%%%%%%%%%%%%%%%%%%%%%%%%%%%%%%
\usepackage{iftex}
% CAREFUL: the order of font includes here is very important!
\ifPDFTeX
  \usepackage[OT1,T1]{fontenc}
  \usepackage[utf8]{inputenc}
  \usepackage{textcomp} % provide euro and other symbols
    \usepackage[p,osf,swashQ]{cochineal}
  \usepackage[cochineal,vvarbb]{newtxmath}
      \usepackage[scale=0.95]{biolinum}
    \usepackage[scale=0.95,varl]{inconsolata}
\else % if luatex or xetex
  \usepackage[scale=0.95,varl]{inconsolata}
  \usepackage{newpxtext}
  \usepackage{mathpazo}
    \usepackage[scale=0.95]{biolinum}
  \fi
\ifLuaTeX
  \usepackage{selnolig}  % disable illegal ligatures
\fi
\IfFileExists{microtype.sty}{% use microtype if available
  \usepackage[]{microtype}
  \UseMicrotypeSet[protrusion]{basicmath} % disable protrusion for tt fonts
}{}

\setlength{\parindent}{0pt}
\setlength{\parskip}{10pt plus 2pt minus 2pt}
\setlength{\emergencystretch}{3em} % prevent overfull lines
\widowpenalty=10000
\clubpenalty=10000
\flushbottom
\allowdisplaybreaks
\sloppy


% CORE PACKAGES %%%%%%%%%%%%%%%%%%%%%%%%%%%%%%%%%%%%%%%%%%%%%%%%%%%%%%%%%%%%
\usepackage[dvipsnames,svgnames,x11names]{xcolor}
\usepackage[lmargin=1.5in,rmargin=1.5in,tmargin=1.2in,bmargin=1.2in]{geometry}
\usepackage[format=plain,
  labelfont={bf,sf,small,singlespacing},
  textfont={sf,small,singlespacing},
  justification=justified,
  margin=0.25in]{caption}

% SECTIONS AND HEADINGS %%%%%%%%%%%%%%%%%%%%%%%%%%%%%%%%%%%%%%%%%%%%%%%%%%%%%%%%
\setcounter{secnumdepth}{4}
\usepackage{sectsty}
\usepackage[compact]{titlesec}
% short title
\makeatletter
\newcommand\@shorttitle{}
\newcommand\shorttitle[1]{\renewcommand\@shorttitle{#1}}
\usepackage{fancyhdr}
\fancyhf{}
\pagestyle{fancy}
\renewcommand{\headrulewidth}{0pt}
\fancyheadoffset{0pt}
%\lhead{\scshape \@shorttitle}
%\rhead{\scshape\today}
\cfoot{\thepage}
\makeatother
% abstract styling
\renewenvironment{abstract}{
  \centerline
  {\large\sffamily\bfseries Abstract}\vspace{-1em}
  \begin{quote}\small
}{
  \end{quote}
}

% PANDOC INCLUDES %%%%%%%%%%%%%%%%%%%%%%%%%%%%%%%%%%%%%%%%%%%%%%%%%%%%%%%%%%%%%%

\providecommand{\tightlist}{%
  \setlength{\itemsep}{0pt}\setlength{\parskip}{0pt}}\usepackage{longtable,booktabs,array}
\usepackage{calc} % for calculating minipage widths
% Correct order of tables after \paragraph or \subparagraph
\usepackage{etoolbox}
\makeatletter
\patchcmd\longtable{\par}{\if@noskipsec\mbox{}\fi\par}{}{}
\makeatother
% Allow footnotes in longtable head/foot
\IfFileExists{footnotehyper.sty}{\usepackage{footnotehyper}}{\usepackage{footnote}}
\makesavenoteenv{longtable}
\usepackage{graphicx}
\makeatletter
\def\maxwidth{\ifdim\Gin@nat@width>\linewidth\linewidth\else\Gin@nat@width\fi}
\def\maxheight{\ifdim\Gin@nat@height>\textheight\textheight\else\Gin@nat@height\fi}
\makeatother
% Scale images if necessary, so that they will not overflow the page
% margins by default, and it is still possible to overwrite the defaults
% using explicit options in \includegraphics[width, height, ...]{}
\setkeys{Gin}{width=\maxwidth,height=\maxheight,keepaspectratio}
% Set default figure placement to htbp
\makeatletter
\def\fps@figure{htbp}
\makeatother
% END PANDOC %%%%%%%%%%%%%%%%%%%%%%%%%%%%%%%%%%%%%%%%%%%%%%%%%%%%%%%%%%%%%%%%%%%

% USER INCLUDES %%%%%%%%%%%%%%%%%%%%%%%%%%%%%%%%%%%%%%%%%%%%%%%%%%%%%%%%%%%%%%%%
% additional LaTeX code for the "preamble" goes here
\makeatletter
\makeatother
\makeatletter
\makeatother
\makeatletter
\@ifpackageloaded{caption}{}{\usepackage{caption}}
\AtBeginDocument{%
\ifdefined\contentsname
  \renewcommand*\contentsname{Table of contents}
\else
  \newcommand\contentsname{Table of contents}
\fi
\ifdefined\listfigurename
  \renewcommand*\listfigurename{List of Figures}
\else
  \newcommand\listfigurename{List of Figures}
\fi
\ifdefined\listtablename
  \renewcommand*\listtablename{List of Tables}
\else
  \newcommand\listtablename{List of Tables}
\fi
\ifdefined\figurename
  \renewcommand*\figurename{Figure}
\else
  \newcommand\figurename{Figure}
\fi
\ifdefined\tablename
  \renewcommand*\tablename{Table}
\else
  \newcommand\tablename{Table}
\fi
}
\@ifpackageloaded{float}{}{\usepackage{float}}
\floatstyle{ruled}
\@ifundefined{c@chapter}{\newfloat{codelisting}{h}{lop}}{\newfloat{codelisting}{h}{lop}[chapter]}
\floatname{codelisting}{Listing}
\newcommand*\listoflistings{\listof{codelisting}{List of Listings}}
\makeatother
\makeatletter
\@ifpackageloaded{caption}{}{\usepackage{caption}}
\@ifpackageloaded{subcaption}{}{\usepackage{subcaption}}
\makeatother
\makeatletter
\@ifpackageloaded{tcolorbox}{}{\usepackage[skins,breakable]{tcolorbox}}
\makeatother
\makeatletter
\@ifundefined{shadecolor}{\definecolor{shadecolor}{rgb}{.97, .97, .97}}
\makeatother
\makeatletter
\makeatother
\makeatletter
\makeatother
% END USER INCLUDES %%%%%%%%%%%%%%%%%%%%%%%%%%%%%%%%%%%%%%%%%%%%%%%%%%%%%%%%%%%%

% BIBLIOGRAPHY %%%%%%%%%%%%%%%%%%%%%%%%%%%%%%%%%%%%%%%%%%%%%%%%%%%%%%%%%%%%%%%%%
\usepackage[]{natbib}
\bibliographystyle{apalike}

% Give it this name so that it works with ::: #refs
\newenvironment{CSLReferences}[2]{
\bibliography{bibliography.bib}
\clearpage
}{}

% LINKS %%%%%%%%%%%%%%%%%%%%%%%%%%%%%%%%%%%%%%%%%%%%%%%%%%%%%%%%%%%%%%%%%%%%%%%%
\usepackage{hyperref}
\usepackage{url}
\hypersetup{
  pdftitle={Reassessing the Impact of Public-Private Partnerships on Political Career Advancement in China},
  pdfauthor={Shuyi Bian},
  colorlinks=true,
  linkcolor={black},
  filecolor={Maroon},
  citecolor={VioletRed4},
  urlcolor={DodgerBlue4},
  pdfcreator={LaTeX via pandoc}}

% TITLE, AUTHOR, DATE %%%%%%%%%%%%%%%%%%%%%%%%%%%%%%%%%%%%%%%%%%%%%%%%%%%%%%%%%%
\title{\sffamily\bfseries\huge\parfillskip=0pt
\rightskip=0pt plus .5\textwidth
\leftskip=0pt plus .5\textwidth
\emergencystretch=.3\textwidth Reassessing the Impact of Public-Private
Partnerships on Political Career Advancement in China}
\shorttitle{Reassessing the Impact of Public-Private Partnerships on Political Career Advancement in China}
\author{\textbf{Shuyi Bian}
 }
\date{}


\begin{document}
\allsectionsfont{\sffamily}

\maketitle

\begin{abstract}
This study interrogates the relationship between Public-Private
Partnership (PPP) investments and mayoral promotions in China, diverging
from Lei and Zhou's assertion of economic performance influencing
political advancement. Employing difference-in-differences and fixed
effects models, the research disaggregates PPP investments by type and
assesses their impact on the promotion of municipal leaders over a
three-year term. The findings reveal no significant effect of PPP
investments on mayoral promotions, with a subtle negative trend for
transportation PPPs, challenging the notion that these investments serve
as a fast track for political ascent. The results suggest a more complex
interplay between infrastructure investment and political incentives
than previously understood, with potential misalignments between
long-term project benefits and short-term political evaluation metrics.
This study prompts a reconsideration of the temporal and evaluative
frameworks used to gauge the political dividends of economic
initiatives, advocating for a broader, more longitudinal perspective in
future research.
\end{abstract}

\ifdefined\Shaded\renewenvironment{Shaded}{\begin{tcolorbox}[borderline west={3pt}{0pt}{shadecolor}, boxrule=0pt, frame hidden, sharp corners, breakable, enhanced, interior hidden]}{\end{tcolorbox}}\fi



% USER BODY %%%%%%%%%%%%%%%%%%%%%%%%%%%%%%%%%%%%%%%%%%%%%%%%%%%%%%%%%%%%%%%%%%%%

\hypertarget{introduction}{%
\section{Introduction}\label{introduction}}

Government plays a crucial role in delivering public goods, with
essential projects like schools, dams, and roads requiring substantial
long-term investment to yield their full benefits. A key issue arises
from the short tenure of local political leaders, which often results in
a lack of incentive to commit to these long-term investments, despite
their eventual benefit to the public. This mismatch between the
short-term focus of political leaders and the long-term public interest
frequently leads to inadequate investment in vital public
infrastructure.\citep{lei2022private}

However, the situation in China presents a notable contrast. Here,
officials display a strong commitment to large-scale, long-term public
projects. This distinctive approach is largely driven by the incentive
structures and career advancement motivations within the Chinese
governance system. Central to this is the performance evaluation system
of the central government, which creates a competitive environment akin
to a `promotion tournament' among officials.\citep{WANG2021106888} In
this system, an official's chances of promotion are heavily influenced
by their economic performance while in office, compared not in absolute
terms but relative to their peers. As a result, officials who perform
well in this competitive setting are more likely to advance in their
careers.\citep{li2005political}

This framework is particularly relevant in understanding China's
economic boom over the past thirty years, which has been significantly
shaped by the central government's focus on GDP growth as a primary
metric. Local officials, under pressure to climb the political
hierarchy, often utilize all the resources at their disposal to
stimulate economic growth in their jurisdictions. This has made economic
expansion a key driver of local government initiatives, particularly in
ambitious, long-term public infrastructure projects like subway systems.
Such projects require significant ongoing investment and have a
substantial impact on local GDP, aligning with the career goals of local
officials in a GDP-centric system. This complex relationship between
infrastructure investment and political career progression in China is
thoroughly examined in the study ``Private Returns to Public Investment:
Political Career Incentives and Infrastructure Investment in China'' by
Zhenhuan Lei and Junlong Aaron Zhou, offering detailed insights into
this intriguing dynamic.

However, the velocity of China's economic expansion has often outpaced
the capacity of government investment, precipitating a significant
fiscal dilemma characterized by mounting government debt and the vast
infrastructural requisites of rapid urbanization. This confluence of
rapid development and financial limitations has engendered a
considerable void in the realm of long-term infrastructural planning and
execution. In response, the Chinese government has robustly endorsed the
model of Public-Private Partnerships (PPPs) as a viable countermeasure
to these economic constraints. Since their initial adoption in the
1980s, exemplified by the Build--Operate--Transfer (BOT) model, PPP have
reflected China's strategic pivot to reconcile the imperatives of urban
development with sustainable fiscal management.\citep{li2023public}
These partnerships have become a linchpin in the financing of diverse
infrastructure projects, crucially facilitating the country's extensive
urbanization process. This PPP-driven approach has proven essential in
enabling the development of critical infrastructure, encompassing
transport and environmental sustainability, amidst the nation's
transformative journey toward urban modernity.

In this study, I extend the discourse initiated by Lei and Zhou by
examining the effects of Public-Private Partnership (PPP) investments on
the promotion odds of municipal leaders within a three-year tenure.
Diverging from Lei and Zhou's conclusions, my analysis segregates PPP
investments into comprehensive categories---`All PPP Project',
`Transportation PPP Project', and `Environmental and Water Conservancy
PPP Project'---and scrutinizes their influence through the prisms of
difference-in-differences (DiD) and fixed effects (FE) models. Contrary
to the hypothesized economic impact-driven promotions postulated by Lei
and Zhou, the results from the DiD models reveal no statistically
significant impact of PPP investments on mayoral promotions, suggesting
that the anticipated economic spillovers may not be as directly
correlated with political progression as previously thought.

Moreover, the FE models, which provide a lens adjusted for city-specific
unobserved variables, support the non-significant findings for `All PPP
Project' and `Environmental and Water conservancy PPP Project'
investments. Notably, the `Transportation PPP Project' investments
demonstrate a minor negative influence on mayoral promotion, alluding to
a potential disconnect between the extended timelines required for
transportation projects to bear fruit and the immediate metrics used for
political ascension. This unexpected negative sign raises questions
about the valuation of investment types by provincial adjudicators
within the temporal bounds of mayoral evaluations. These findings, when
juxtaposed against Lei and Zhou's work, suggest a more intricate and
less direct relationship between PPP investment types and their
political rewards. It appears that the three-year window may not capture
the full economic ramifications of such investments, prompting a need
for further exploration into the longitudinal impacts and a broader
understanding of the incentives at play in provincial officials'
assessments of mayoral efficacy.

\hypertarget{data-analysis}{%
\section{Data \& Analysis}\label{data-analysis}}

In Zhenhuan Lei and Junlong Aaron Zhou's study on subway approvals in
Chinese cities, they employed a comprehensive data collection
methodology utilizing various sources. The initial data set, focusing on
subway approvals, was primarily sourced from the annual reports of the
China Association of Metros, identifying cities with existing or
under-construction subway systems as of 2017.

In their study, Lei and Zhou meticulously analyzed the career dynamics
of city mayors in China, drawing on the CCER Official Dataset and the
Chinese Political Elite Database. These resources provided insights into
the mayors' promotions, political connections, and career histories.
Additionally, they utilized the China City Statistical Yearbook and the
China Urban Construction Statistical Yearbook for city-level
socio-economic and infrastructural data. Focusing on 265
prefecture-level cities from 2003 to 2016, while excluding megacities
for consistency, they formed a comprehensive city-year panel data set.

\hypertarget{tbl-main}{}
\begin{longtable}[]{@{}l@{}}
\caption{\label{tbl-main}Replication of \citet{lei2022private} Summary
Statistics Table}\tabularnewline
\toprule\noalign{}
\endfirsthead
\endhead
\bottomrule\noalign{}
\endlastfoot
\includegraphics{figures/SummaryTable1.jpg} \\
\end{longtable}

Table 1 reveals insightful trends: a 41\% average promotion rate for
mayors over three years suggests a dynamic environment for career
advancement, with political connections appearing less influential than
expected. The data also highlights the mayors' average age of 50,
pointing to a preference for experienced leadership. Additionally, the
diversity in city populations, GDP, and infrastructure investments
underscores the varying economic landscapes of these cities.
Particularly notable is the rarity of subway approvals, with an average
of only 0.04 mayors achieving this feat, illustrating the uniqueness and
potential significance of such accomplishments in the context of urban
development and political careers.

To assess the influence of subway approvals on mayoral promotions in
China, Lei and Zhou utilize a generalized difference-in-differences
(DID) approach as their baseline identification strategy.

\[Promotion_{it} = \beta_0 + \beta_1 Approval_{it} + \gamma X_{it-1} + \theta_i + \pi_t + \varepsilon_{it}\]

In the DiD model, the dependent variable, Promotion\_\{it\}, signifies
whether a mayor advances to higher political office within three years.
The key explanatory variable, Approval\_\{it\}, denotes subway project
approval, with mayors identified during the pivotal approval-seeking
phase. Lei and Zhou control the city-specific trends and annual effects
to isolate the impact and they focusing on the positive relationship
indicated by coefficient \(\beta_1\), which suggests that subway
approvals are significantly associated with increased chances of
promotion.

Moreover, to test ``parallel trends assumption'' and demonstrate the
dynamic effect of subway approval on a mayor's promotion. They use a new
model to analyze approvals across different years, with
Promotion\_\{it\} reflecting the promotion outcome.

\[Promotion_{it} = \sum_{\substack{\gamma=-4 \\ \gamma \neq +1}}^{+5} \beta_{\gamma} Approval_{i(t+\gamma)} + \omega X_{it-1} + \theta_i + \pi_t + \varepsilon_{it}.\]
The analysis hinges on the condition that approval impact coefficients
(\(\beta_\gamma\)) for non-base years should not differ significantly
from zero, ensuring the validity of the DID approach.

\begin{figure}[tbp]

{\centering \includegraphics[width=0.8\textwidth,height=\textheight]{figures/Lead_Lag_Figure1.jpg}

}

\caption{\label{fig-example}Dynamic effects of subway approvals on mayor
promotion. Each circle indicates a point estimate for the effect of
subway approval, and vertical bars are the 90\% and 95\% confidence
intervals. Negative num- bers on the horizontal axis refer to the years
before a city receives subway approval; positive numbers indicate the
years since the city receives subway approval. We omit the year before
the city obtains subway approval as a baseline. All coefficients should
be inter- preted in comparison with this baseline year.@lei2022private}

\end{figure}

Figure 1 delineates the dynamic effects of subway approvals, where the
period prior to approval serves as a baseline, exhibiting no significant
impact on mayoral promotion prospects. This baseline adherence fortifies
the assumption that, absent the approval, promotion trends would remain
unchanged. The figure then reveals a marked positive shift in the
likelihood of promotion post-approval, affirming that the event of
subway approval acts as a catalyst for mayoral career advancement.
Crucially, the effect is sustained only during the tenure of the mayors
who secured the approval, as subsequent mayors do not benefit from this
legacy. This temporal specificity is visually captured in the figure,
where point estimates post-approval rise significantly above the
baseline, with confidence intervals excluding the zero-effect threshold.
Such findings underscore the individual achievement linked to the subway
approval and its substantial, albeit time-bound, influence on promotion
within the mayoral tenure, thereby substantiating the precision and
validity of Lei and Zhou's analytical approach in their study.

\hypertarget{results}{%
\section{Results}\label{results}}

Table 2, which outlines the core results of their multivariate
regression analysis, confirms a statistically significant positive
correlation between subway approvals and mayoral promotions across
various model specifications.

\hypertarget{tbl-main}{}
\begin{longtable}[]{@{}l@{}}
\caption{\label{tbl-main}Replication of \citet{lei2022private}
Difference in Difference design}\tabularnewline
\toprule\noalign{}
\endfirsthead
\endhead
\bottomrule\noalign{}
\endlastfoot
\includegraphics{figures/AnalysisTable2.png} \\
\end{longtable}

The baseline model (Column 1) controls for city and year fixed effects
and indicates that subway approval is associated with a 25.1\% increase
in the likelihood of mayoral promotion. This relationship persists even
after progressively adding layers of complexity to the model: mayoral
characteristics such as age, gender, ethnicity, education, and political
connections (Column 2); city-level economic indicators like population,
GDP size, and growth rate (Column 3); and province-year fixed effects to
account for broader regional and temporal trends (Column 4).

Each model reinforces the robust positive effect of subway approval on
mayoral promotion, despite the introduction of additional controls. The
analysis, conducted with rigorous statistical methods, employs standard
errors clustered at the city level to correct for within-group
correlation, enhancing the credibility of the findings. With an average
promotion rate of 42.6\% to 43.2\% across the models, and a sample size
ranging from 3,071 to 3,647 observations, Lei and Zhou provide
compelling evidence that securing subway approval significantly boosts a
mayor's promotion prospects, independent of a comprehensive array of
personal and city-level factors. This nuanced approach to modeling and
the consistent significance across various specifications underscore the
conclusion that subway approvals play a substantial role in shaping the
career trajectories of city mayors in China.

\hypertarget{ppp-investments-on-mayoral-promotion}{%
\section{PPP investments on mayoral
promotion}\label{ppp-investments-on-mayoral-promotion}}

I expand upon Lei \& Zhou's study by using both fixed effect model and
difference-in-differences model to test the effects of Public-Private
Partnership (PPP) investments on the promotion odds of municipal leaders
within a three-year tenure. The PPP data matches PPP contracts with
Chinese mayors who served from 2010 to 2017.\citep{li2023public}

\hypertarget{tbl-main}{}
\begin{longtable}[]{@{}l@{}}
\caption{\label{tbl-main}PPP Projects and Mayoral
Promotion}\tabularnewline
\toprule\noalign{}
\endfirsthead
\endhead
\bottomrule\noalign{}
\endlastfoot
\includegraphics{figures/PPPTable3.png} \\
\end{longtable}

The new data examines the effect of Public-Private Partnership (PPP)
investments on mayoral promotion within three years, disaggregated by
type of investment, to test the hypothesis posited by Lei and Zhou that
economic impact drives the promotion of mayors. The types of PPP
investment analyzed are `All PPP', `Transportation PPP', and
`Environmental PPP', with each category assessed using both
difference-in-differences (DiD) and fixed effects (FE) models.

Table 3, which the results show that for `All PPP' investments,
`Transportation PPP', and `Environmental PPP', there is no statistically
significant effect on the likelihood of mayoral promotion, as indicated
by the coefficients that are effectively zero across all DiD models.
These findings suggest that, contrary to Lei and Zhou's claim, not all
types of investment yield a perceptible economic spillover that results
in mayoral promotion, at least not within the three-year window
examined. This could imply that provincial party officials may not
perceive all PPP investments as equally valuable or that the economic
benefits of such investments do not materialize quickly enough to
influence mayoral promotions within the short timeframe considered.

The FE models, which control for unobserved city-level heterogeneity,
also report no significant effects for `All PPP Project', `Environmental
and Water Conservancy PPP Project', with `Transportation PPP Project'
showing a small negative effect. This negative coefficient in the
`Transportation PPP Project' FE model could be due to various reasons,
such as the long gestation period of transportation projects not
aligning with the short-term assessment period for promotion, or that
such investments are not valued as highly by provincial officials when
evaluating mayoral performance.

In sum, the PPP data do not support the notion that investment type
consistently correlates with mayoral promotion, as posited by Lei and
Zhou. These findings could indicate a more complex relationship between
investment types, economic spillovers, and political rewards, or they
could reflect that the time frame for evaluating the economic impacts of
these investments is insufficient. Further research might explore
longer-term effects, different kinds of economic benefits, or
alternative incentives for provincial officials when assessing mayoral
performance

\hypertarget{conclution}{%
\section{Conclution}\label{conclution}}

This study extends the foundational research of Lei and Zhou by
examining the relationship between Public-Private Partnership (PPP)
investments and the promotion trajectories of municipal leaders in China
within the typical three-year mayoral tenure. The analysis, employing
both difference-in-differences and fixed effects models, explores
various categories of PPP investments, namely `All PPP', `Transportation
PPP', and `Environmental PPP'. Contrary to Lei and Zhou's findings that
linked economic achievements to promotions, this study reveals no
significant effect of PPP investments on the likelihood of mayoral
promotion across the board. The results, particularly for
`Transportation PPP' investments, indicate a minor negative influence,
suggesting that these investments might not align with the short-term
metrics used for assessing political advancement, or perhaps provincial
officials value such investments differently.

The absence of a significant positive correlation between PPP
investments and mayoral promotions within the three-year window
questions the assumed direct linkage between economic impact and
political ascent posited in prior studies. This finding points to a more
nuanced and potentially less straightforward relationship. The study's
limitations, such as the constraints of the three-year observational
window and the varying timelines required for PPP projects to yield
tangible economic benefits, suggest that a longer-term perspective may
be necessary to fully understand the dynamics at play. Future research
could profitably investigate the long-term economic effects of PPP
investments and consider alternative evaluation metrics used by
provincial officials. The implications of this study are significant, as
they invite policymakers and scholars to reconsider the incentives that
drive the commitment of local officials to public infrastructure
projects and to explore new avenues for aligning long-term public
benefits with political career incentives.

\hypertarget{refs}{}

\begin{CSLReferences}{0}{0}\end{CSLReferences}


% END BODY %%%%%%%%%%%%%%%%%%%%%%%%%%%%%%%%%%%%%%%%%%%%%%%%%%%%%%%%%%%%%%%%%%%%%



\end{document}
